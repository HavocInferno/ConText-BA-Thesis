% !TeX root = ../main.tex
% Add the above to each chapter to make compiling the PDF easier in some editors.

\chapter{Research}\label{chapter:research}

\section{Inspiration}
The concept of text adventures, choice based games or gamebooks is by far not new, and dates back long before electrical computers were even invented. They are a type of game that work almost inherently well on any type of device capable of displaying text and accepting input, as that in its most basic form is all they need. However, the question with every type of device remains how that is done intuitively and most true to the user's preferred consumption method with that device. With modern smartphones and tablets, the most popular input methods are tapping and swiping, and extensions of those to perform more complex actions. For a text adventure, this is contrast to the classic physical typing on a keyboard or even the pointing and clicking with a mouse. As such, it is clear a mobile text adventure ideally adapts and adopts tapping and swiping. In the spirit of the aforementioned game genres, that means swiping through texts or other forms of content and tapping items, replies or similar. 
Another aspect of these types of games has always been the deliberate lack of visual stimuli. Much like the classic understanding of books, they aim to elicit purely imaginary experiences. The user should mentally picture the events, draw their own version of the world, and not be bound to an artist's specific interpretation. 
One game in particular that we think captures the essence of these two sides very well is the Lifeline series of games developed by Three Minute Games in 2015 and 2015 for Apple iOS and Google Android. The player is confronted with an at the beginning unknown world and one character. Descriptions of the world and events are delivered through the eyes of that character, and all the player can sometimes do is choose between two replies to a prompt by the character. Despite this simplicity, through the course of the game, the player envisions the entire world, the looks, the events and everything as they see fit. At the same time, the game is simple enough to be played even while being mobile with a smartphone in hand and does not require constant attention. 
This sparked the question about what options are available nowadays for creating this type of game. As looking at related work showed, the readily available options are often lackluster, but also provide valuable insight into what features need to be combined into one tool to provide satisfying overall functionality to both inexperienced and experienced user groups.

\section{Related work}