% !TeX root = ../main.tex
% Add the above to each chapter to make compiling the PDF easier in some editors.

\chapter{Accessibility and user perception}\label{chapter:userstudy}
In order to determine the quality of the ConText framework, the accessibility for both experienced and inexperienced users and the quality of the general user experience, user studies needed to be conducted. 

\section{Study procedure}
The quality evaluation for ConText consisted of two consecutive user studies, the procedure for which was essentially identical. Participants were selected and invited to take part in the study, with a small number planned for the first study to get an initial idea of the user perception and qualitative direction the framework was headed at that point in development and a larger number of participants intended for the second study that would evaluate the progress and changes made between both studies and provide more differentiated  feedback. Due to time constraints and unfortunate circumstances, however, the second user study did not work out as intended and only returned very lackluster results. A more elaborate rundown of the issue is provided in section 4.3.

In order to ensure the participants would see all relevant parts of the framework, they were given an orientation sheet and a tutorial to guide them through the creation of a simple game. Additionally included was a documentation explaining remaining and individual features of the framework, followed by a survey sheet - either as a PDF or a Google Forms form to be filled out online - to rate their experience and give feedback on certain aspects.
All four documents are included in Appendix section A.1.
\paragraph{Orientation} The orientation sheet presents an overview of what the study task is and what each of the other documents is intended to be as well as informs the participant on what the study and its results will be used for. The participant is guaranteed that their creation will not be used unless explicitly permitted and that the results of the study will only be used and evaluated for the thesis.
\paragraph{Tutorial} The tutorial is a two-part guide through the framework. 
Part 1 explains the installation and setup process, from installing Unity through importing the framework files to setting up the main screen. 
Part 2 is a step by step walkthrough to creating a sample game with the framework, including creating and configuring characters, creating, configuring and linking different types of modules and finally how to test the game within the editor. A final section explains the process of importing own files into Unity.
\paragraph{Documentation} The documentation details the various aspects of the framework in order to possibly explain functionality not covered in the tutorial but of interest to the user in further usage. It covers the module system and its structure, the managers and their functionality, the characters and their purpose, the UI settings and theirs, the custom inspectors and their heritage setup as well as what to watch out for when creating custom modules.
\paragraph{Survey} The survey sheet queries the participant on their experience with the framework and opinions and feedback on certain aspects. The participant is first to describe the computer system they are using for development and their preexisting skills regarding game development and story writing. The further parts of the survey prompt the participant to rate and if possible explain their rating for the intuitiveness, complexity and arrangement of the interface, the perceived performance and responsiveness of the tools, the quality and detail of the documentation and their overall experience using the framework. 
\paragraph{Study files} Accompanying the four above documents the participants get access to a Unity Package containing all technical parts of the framework such as the code base and the initially set up scene, while the last file is a Unity layout file. 

\section{First user study}
The first user study had five participants out of seven invited. All participants were given four days as well as all aforementioned files to create a small game, fill out the survey and send their feedback. The statistics and results of the study are discussed in the following paragraphs, sorted by their order within the survey sheet.
\subsection{Evaluation and results (del)}
\paragraph{Basics} TODO.
\paragraph{Interface} TODO.
\paragraph{Performance} TODO.
\paragraph{Documentation/Tutorial} TODO.
\paragraph{General impression} TODO.
\paragraph{Additional} TODO.

\section{Second user study}
TODO.
\subsection{Changes (del)}
\subsection{Evaluation and results (del)}